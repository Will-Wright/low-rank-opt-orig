%--------------
\newtheorem{theorem}{Theorem}[section]
\newtheorem{lemma}[theorem]{Lemma}
\newtheorem{corollary}[theorem]{Corollary}
\newtheorem{proposition}[theorem]{Proposition}
\newtheorem{definition}[theorem]{Definition}
\newtheorem{conjecture}[theorem]{Conjecture}

\newtheorem{remark}[subsection]{Remark}
\newtheorem{remarks}[subsection]{Remarks}
\newtheorem{example}[subsection]{Example}
\floatname{algorithm}{Listing}



%% commenting
% Affect margins:
%\setlength{\marginparwidth}{1.2in}
\setlength{\marginparwidth}{.8in}
\let\oldmarginpar\marginpar
\renewcommand\marginpar[1]{\-\oldmarginpar[\raggedleft\footnotesize #1]%
{\raggedright\footnotesize #1}}

% For making comments in the draft:
% Stephen (SRB)'s comments:
\newcommand{\srbmargin}[1]{\marginpar{\raggedright\scriptsize\textcolor{red}{SRB: #1}}}
\newcommand{\tfocsMargin}[2][0]{\marginpar{\vspace{#1 em} \rule{.2mm}{1em} \texttt{#2} }}
\setlength{\marginparpush}{1em}  % min. vertical distance between margin comments
\setlength{\marginparsep}{1ex} % horizontal distance from main text to start of margin
% define an underscore character:
\def\us{\char`\_}

\newcommand{\srb}[1]{\textcolor{red}{SRB: #1}}


%% MCG's definitions
\newcommand\thalf{{\textstyle\frac{1}{2}}}
\newcommand\DSA{$\texttt{DS}^2$\xspace}
\newcommand\dom{\operatorname{\textrm{dom}}}
\newcommand\Int{\operatorname{\textrm{Int}}}
\newcommand{\bmat}[1]{\begin{bmatrix} #1 \end{bmatrix}}

%% TT's definitions
\newcommand\tr{{{\operatorname{trace}}}}
\newcommand{\eps}{\varepsilon}

% SRB's definitions
\newcommand{\AAA}{\ensuremath{\mathcal{A}}}   % generic linear operator
\newcommand{\cA}{\ensuremath{\mathcal{A}}}    % generic linear operator
\newcommand{\K}{\ensuremath{\mathcal{K}}}     % cone
\newcommand{\cK}{\ensuremath{\mathcal{K}}}    % cone
\newcommand{\proj}{\ensuremath{\mathcal{P}}}  % Projection
\renewcommand{\P}{\operatorname{\mathbb{P}}}

\newcommand{\lag}{\ensuremath{\mathcal{L}}}   % Lagrangian
\renewcommand{\L}{{\mathcal L}}

\newcommand{\order}{\mathcal{O}}              % big O notation
\DeclareMathOperator*{\argmax}{argmax}        % puts subscripts in the right place
\DeclareMathOperator*{\argmin}{argmin}
\DeclareMathOperator*{\minimize}{minimize}
\DeclareMathOperator*{\maximize}{maximize}
\newcommand{\st}{\ensuremath{\;\text{such that}\;}}
\newcommand{\gs}{g_\text{sm}}             % smooth part of dual objective

% shortcuts
\newcommand{\za}{z^{(1)}}
\newcommand{\zb}{z^{(2)}}

% to get the ones vector to look nice (without using the bbold package)
\newcommand{\bbfamily}{\fontencoding{U}\fontfamily{bbold}\selectfont}
\newcommand{\textbb}[1]{{\bbfamily#1}}
\DeclareMathAlphabet{\mathbbb}{U}{bbold}{m}{n}
\newcommand{\ones}{\mathbbb 1}                % ones vector 

\DeclareMathOperator{\vect}{vec}            % vec(X) = X(:) in matlab notation
\DeclareMathOperator{\mat}{mat}             % mat(x) = reshape(x,N,N)

\DeclareMathOperator{\prox}{prox}             % mat(x) = reshape(x,N,N)

%\newcommand{\defeq}{\mathrel{\mathop:}=}      % for definitions, e.g. z := y + 3
%\newcommand{\defeq}{\triangleq}               %   another alternative
%\newcommand{\defeq}{\equiv}                   %   another alternative
\newcommand{\defeq}{\stackrel{\text{\tiny def}}{=}}  %   another alternative
%\newcommand{\defeq}{\stackrel{\text{\tiny def}}{\hbox{\equalsfill}}}  % another alternative, doesn't work


%--------------
% EJC's macros
\newcommand{\R}{\mathbb{R}}
\newcommand{\C}{\mathbb{C}}
\newcommand{\Z}{\mathbb{Z}}
\newcommand{\<}{\langle}
\renewcommand{\>}{\rangle}
\newcommand{\Var}{\textrm{Var}}
\newcommand{\goto}{\rightarrow}
\newcommand{\E}{\operatorname{\mathbb{E}}}
\newcommand{\norm}[1]{{\left\lVert{#1}\right\rVert}}
\newcommand{\col}{\textrm{col}}

\newcommand{\e}{\mathrm{e}}
\renewcommand{\i}{\imath}

% macros for the outline
\newcommand{\todo}{{\bf \textcolor{red}{TODO} }}
\newcommand{\TODO}[1]{{\bf TODO: #1}}
\newcommand{\red}{\textcolor{red}}
\newcommand{\note}[1]{{\bf [{\em Note:} #1]}}

% Specific commands
\newcommand{\ud}{\underline{\delta}}


% Linear algebra macros
%\newcommand{\vct}[1]{\bm{#1}}
%\newcommand{\mtx}[1]{\bm{#1}}
\newcommand{\vct}[1]{{#1}}
\newcommand{\mtx}[1]{{#1}}
%\newcommand{\mtx}[1]{\mathsfsl{#1}}
\renewcommand{\vec}[1]{{\boldsymbol{#1}}}

%\newcommand{\T}{*}                           % (see also \transp, \adj below)
\newcommand{\T}{T}                            % for the adjoint/transpose
\newcommand{\transp}{T}
\newcommand{\adj}{*}
\newcommand{\psinv}{\dagger}

\newcommand{\sgn}{\textrm{sgn}}
\newcommand{\sign}{\textrm{sgn}}
\newcommand{\shr}{\operatorname{shrink}}
\newcommand{\shrink}{\operatorname{shrink}}
\newcommand{\trunc}{\operatorname{trunc}}

\newcommand{\lspan}[1]{\operatorname{span}{#1}}
\newcommand{\range}{\operatorname{range}}
\newcommand{\colspan}{\operatorname{colspan}}
\newcommand{\rank}{\operatorname{rank}}
\newcommand{\diag}{\operatorname{diag}}
\newcommand{\trace}{\operatorname{trace}}
\newcommand{\supp}[1]{\operatorname{supp}(#1)}

\newcommand{\smax}{\sigma_{\max}}
\newcommand{\smin}{\sigma_{\min}}

\newcommand{\restrict}[1]{\big\vert_{#1}}

\newcommand{\Id}{\text{\em I}}
\newcommand{\OpId}{\mathcal{I}}


\numberwithin{equation}{section}

\def \endprf{\hfill {\vrule height6pt width6pt depth0pt}\medskip}
\newenvironment{proof}{\noindent {\bf Proof} }{\endprf\par}


\newcommand{\qed}{{\unskip\nobreak\hfil\penalty50\hskip2em\vadjust{}
           \nobreak\hfil$\Box$\parfillskip=0pt\finalhyphendemerits=0\par}}


\newcommand{\iprod}[2]{\left\langle #1 , #2 \right\rangle}
\newcommand{\iprodMed}[2]{\Bigl\langle #1 , #2 \Bigr\rangle}



\pagestyle{plain}
